\documentclass[a4paper,man,natbib]{apa6}

\usepackage[english]{babel}
\usepackage[utf8x]{inputenc}
\usepackage{amsmath}
\usepackage{graphicx}
\usepackage[colorinlistoftodos]{todonotes}

\title{A Review of the Movie \textit{Apocalypse Now}}
\shorttitle{A Review of the Movie \textit{Apocalypse Now}}
\author{Samuel Pilla}
\affiliation{Missouri University of Science and Technology}

\abstract{This paper will explore the applications of Abnormal Psychology in the film \textit{Apocalypse Now}, including correct and incorrect aspects of symptoms, diagnosis, and other disorders that are depicted in the movie. Using the DSM-V as the source for determining the accuracy of disorders, the characters in the film, primarily Captain Benjamin L. Willard and Colonel Walter E. Kurtz, will be thoroughly examined to outline which disorder, or disorders, they exhibit and which the film may inaccurately depict. Because the film is based in the Vietnam War, the perspective when analysing the characters will be shifted to fit the setting, thus potentially causing a diagnosis to be different than the diagnosis of a “normal” life experience.}

\begin{document}
\maketitle

\section{Introduction}

\textit{Apocalypse Now} focuses on Captain Benjamin L. Willard and what he describes as his last mission during the Vietnam War. The film opens with mental images and sounds of the war, hearing the fan as a chopper and watching tree lines being incinerated by air strikes. When he finally snaps out of his trance, he realizes he is not in the jungle, where he longs to be, describing his brief life back home after his first tour boring, dull, and undesirable. He had divorced his wife, and only wanted to be back in action. While in the hotel room, he slowly falls into madness. Drinking, smoking, wild dancing, punching glass, and other self-inflicted wounds finally lead up to him being called upon for a very particular mission, one he describes as "the last he will ever need".

After being assisted in getting cleaned up and presentable, he meets with some officers to find out about the mission: he must assassinate the rogue Colonel Kurtz, who has gone to Cambodia and built his own little empire, being praised as a god, and leading with ruthless authority. Being described as insane and off the rails, Willard is encouraged to kill with extreme prejudice.

As Willard begins his decent towards Kurtz, he reflects on his own actions, on how many "Charlies", or Vietcong, he has killed, and begins to wonder what happened to Kurtz. He meets up with the Navy crew known as the PBR, consisting of the captain Chief, and three others, Chef, Clean, and Lance. They is assigned to take Willard where he needs to go, however, being that the mission is top secret, the crew is very hesitant.

The crew boats towards the Nung River, being the river that Willard must take to reach Kurtz. Before reaching the river, they rendezvous with Lieutenant Colonel Bill Kilgore, who is in charge a of air squadron. They discuss getting to the Nung River, and are met with great resistance until Kilgore discovers Lance is a professional surfer. Being a surfing fan himself, and discovering one of the main entrances to the Nung River has 6 foot waves, he agrees to assist. When arriving at the mouth of the river, Kilgore shows his true colors, blasting "Ride of the Valkyries" while raining death onto the village below. During the siege, Kilgore demands two of his men test the waters for surfing, while mortars are bombarding around them. Eventually, a napalm strike on the jungle tree line ends most of the fighting, but begins to end the good surf. Kilgore pleads with Lance to stick around instead of continuing on the Nung River, but Willard craftily get the PBR crew out, and even manages to steal Kilgore's favorite surf board.

While hiding out from Kilgore's search parties, Chef and Willard wander into the jungle in search of mango's, only to run into a tiger. This leads Chef to curse and swear he is done with the war, demanding to leave. However, he is calmed down while the crew continues, eventually running into Hau Phat, a resupply camp with a show from the Playboy bunnies. Given free tickets, the crew is gathered in with hundreds of other US soldiers, all raving maniacally to have the women sign their photos, touch them, and eventually swarm the stage, forcing the women to leave.

Although only temporarily sated with a distraction from the war and the mission, the crew continues up the river further towards Kurtz. They eventually reach a camp dring a torrential downpour. While trying to find the commanding officer of the camp, Willard stumbles upon the "agent" of the bunnies, and strikes a deal for his crew to spend time with them. While Chef and Lance spend time with two of the bunnies, it becomes apparent they are not entirely willing to be there. \textit{Lance realized that "his bunny" has done things she was very against, but still did them. Eventually, they find a dead body and he comforts her.} Chef almost entirely ignores "his bunny" as she talks about her love for birds, undressing and posing her to be a mirror of his favorite picture. Eventually, he begins to act like a bird, causing "his bunny" to become sexually aroused.

After leaving the Medevac camp, Clean complains about not being able to spend time with either bunny, and gets shrugged off by both Chef and Lance. Continuing down the river, they stop a trading boat for a routine check for weapons or Vietcong support. Willard demands they do not, but Chief is insistent. While checking the boat, nothing shows up until a yellow barrel set everyone off guard, causing a Vietnamese woman to yell not to check it. Clean then loses it, and begins firing the machine gun at the trade boat, leading Lance to fire maniacally as well, killing all but the Vietnamese woman. All they find is a puppy, which Lance takes as his own. While Chief tells Chef to bring the woman over to bring to a hospital, Willard kills her, reinforcing his previous statement to not stop.

All basically silent moving on, they arrive at the last US station, the Do Long Bridge, a bridge constantly being destroyed at night, and rebuilt in the day. Willard and Lance begin the search for the commanding officer, slowly realizing the insanity and chaos the men at the bridge have fallen into. Watching the bridge slowly collapse from Vietcong fire, they leave with no fuel or real assistance, and continue the increasingly chaotic path down to Kurtz.

Out of nowhere, a barrage of fire consumes the boat, cause panic and return fire. In the heat of the barrage, Clean is downed, and only after, does the rest of the crew. Devastated by the death of their 17 year old comrade, the sullen remaining crew is stopped in the fog by a mysterious by French troops. After speaking with them, they come to an agreement, letting the US crew bury and respectfully say goodbye to Clean. After burying Clean, Willard finds himself at dinner, finding out about the French clan's plantation and how determined they are to stay. After losing World War 1 and 2, the these French demand victory even to death, becoming enraged with Willard and the US, pointing out how they may win every battle, but have no chance to win the war. Nothing can be gained there for the US, and thus drive the families to retire for the evening.

One particular women, Roxanne, stays, and begins connecting with Willard. Roxanne lost her husband, and tells Willard about how she would tell him "There are two of you don't you see? One that kills. One that loves." Willard begins to realize he is exactly that, a man that kills, and one that loves. However his love had been lost, lost when he left his home, became enthralled in the war and lost touch until now, during his time with Roxanne.

On the final stretch to Kurtz, another barrage floods the boat again, except this time of arrows. Chief orders return fire, but after Willard realizes they are just toy arrows, he demands a cease fire. When Chief becomes enraged with the desire of power that Willard has, he turns his back and a spear impales him. As Chief chokes on his own blood, he begins to attempt to choke out Willard, but Willard responds and smothers Chief.

Having lost two men, the crew, only Chef, Lance, and Willard remaining, finally arrive at Kurtz's base. Greeted by a single American photographer and hundreds of Kurtz's followers. Slowly realizing the madness that Kurtz has descended into, ruthlessly murdering people who attempt to defy his reign. While exploring the camp, they come across the practically brainwashed Colby, the last man sent to assassinate Kurtz. As the new servant stares down Willard and Chef, they realize just the extremity these people will go for Kurtz.

They return again, this time Willard and Lance, to talk to Kurtz, and telling Chef to call an airstrike if they do not return. While finally meeting Kurtz in person, Willard is captured and imprisoned, while Lance, having totally lost touch with his surfing reality, becomes one of Kurtz's men. After being tied up, Kurtz returns, dropping the head of Chef, not having gotten the airstrike called in, onto Willard's lap. Willard becomes distraught, realizing the mistakes he has made.

Eventually he is set free, under prohibition, wandering the colony Kurtz has built. While under watch, Kurtz lectures Willard on the reality of war and humanity, reflecting the French plantation words that US can not win with the dedication of the Vietcong.

That night the colony throws a celebration, while Willard sneaks into Kurtz quarters. As Kurtz reads off quotes to send to the US army, Willard attacks him with a machete, eventually forcing Kurtz to the ground. As Kurtz breaths his final breaths, he utters the words "The horror...the horror...". As Willard leaves from Kurtz's quarters, praised by the now aware colony as the new god, and grabs Lance to leave. Finishing with the two leaving the colony  in silence, it begs the question of who is really insane in war, particularly on the can not be won.

\section{Some Application to Abnormal Psychology}
\label{sec:examples}

\subsection{Aspects of the Movie that were Correct}

Use section and subsection commands to organize your document. \LaTeX{} handles all the formatting and numbering automatically. Use ref and label commands for cross-references.

\subsection{Aspects of the Movie that were Incorrect}

You can add inline TODO comments with the todonotes package, like this:
\todo[inline, color=green!40]{This is an inline comment.}

\subsection{Conclusion}

LaTeX automatically generates a bibliography in the APA style from your .bib file. The citep command generates a formatted citation in parentheses \citep{Lamport1986}. The cite command generates one without parentheses. LaTeX was first discovered by \cite{Lamport1986}.

\pagebreak
\maketitle

\subsection{References}
Coppola, F. F. (Producer), \& Coppola, F. F. (Director). (1979). \textit{Apocalypse Now} [Motion picture]. United States: Omni Zoetrope Studios.

American Psychiatric Association. (2013). Diagnostic and statistical manual of mental disorders (5th ed.). Washington, DC: Author.

\end{document}

%
% Please see the package documentation for more information
% on the APA6 document class:
%
% http://www.ctan.org/pkg/apa6
%