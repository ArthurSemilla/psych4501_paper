\documentclass[a4paper,man,natbib]{apa6}

\usepackage[english]{babel}
\usepackage[utf8x]{inputenc}
\usepackage{amsmath}
\usepackage{graphicx}
\usepackage[colorinlistoftodos]{todonotes}

\title{A Review of the Movie \textit{Apocalypse Now}}
\shorttitle{A Review of the Movie \textit{Apocalypse Now}}
\author{Samuel Pilla}
\affiliation{Missouri University of Science and Technology}

\abstract{This paper will explore the applications of Abnormal Psychology in the film \textit{Apocalypse Now}, including correct and incorrect aspects of symptoms, diagnosis, and other disorders that are depicted in the movie. Using the DSM-V as the source for determining the accuracy of disorders, the characters in the film, primarily Captain Benjamin L. Willard and Colonel Walter E. Kurtz, will be thoroughly examined to outline which disorder, or disorders, they exhibit and which the film may inaccurately depict. Because the film is based in the Vietnam War, the perspective when analysing the characters will be shifted to fit the setting, thus potentially causing a diagnosis to be different than the diagnosis of a “normal” life experience.}

\begin{document}
\maketitle

\section{Introduction}

\textit{Apocalypse Now} focuses on Captain Benjamin L. Willard and what he describes as his last mission during the Vietnam War. The film opens with mental images and sounds of the war, hearing the fan as a chopper and watching tree lines being incinerated by air strikes. When he finally snaps out of his trance, he realizes he is not in the jungle, where he longs to be, describing his brief life back home after his first tour boring, dull, and undesirable. He had divorced his wife, and only wanted to be back in action. While in the hotel room, he slowly falls into madness. Drinking, smoking, wild dancing, punching glass, and other self-inflicted wounds finally lead up to him being called upon for a very particular mission, one he describes as "the last he will ever need".

After being assisted in getting cleaned up and presentable, he meets with some officers to find out about the mission: he must assassinate the rogue Colonel Kurtz, who has gone to Cambodia and built his own little empire, being praised as a god, and leading with ruthless authority. Being described as insane and off the rails, Willard is encouraged to kill with extreme prejudice.

As Willard begins his decent towards Kurtz, he reflects on his own actions, on how many "Charlies", or Vietcong, he has killed, and begins to wonder what happened to Kurtz. He meets up with the Navy crew known as the PBR, consisting of the captain Chief, and three others, Chef, Clean, and Lance. They is assigned to take Willard where he needs to go, however, being that the mission is top secret, the crew is very hesitant.

The crew boats towards the Nung River, being the river that Willard must take to reach Kurtz. Before reaching the river, they rendezvous with Lieutenant Colonel Bill Kilgore, who is in charge a of air squadron. They discuss getting to the Nung River, and are met with great resistance until Kilgore discovers Lance is a professional surfer. Being a surfing fan himself, and discovering one of the main entrances to the Nung River has 6 foot waves, he agrees to assist. When arriving at the mouth of the river, Kilgore shows his true colors, blasting "Ride of the Valkyries" while raining death onto the village below. During the siege, Kilgore demands two of his men test the waters for surfing, while mortars are bombarding around them. Eventually, a napalm strike on the jungle tree line ends most of the fighting, but begins to end the good surf. Kilgore pleads with Lance to stick around instead of continuing on the Nung River, but Willard craftily get the PBR crew out, and even manages to steal Kilgore's favorite surf board.

While hiding out from Kilgore's search parties, Chef and Willard wander into the jungle in search of mango's, only to run into a tiger. This leads Chef to curse and swear he is done with the war, demanding to leave. However, he is calmed down while the crew continues, eventually running into Hau Phat, a resupply camp with a show from the Playboy bunnies. Given free tickets, the crew is gathered in with hundreds of other US soldiers, all raving maniacally to have the women sign their photos, touch them, and eventually swarm the stage, forcing the women to leave.



\section{Some Application to Abnormal Psychology}
\label{sec:examples}

\subsection{Aspects of the Movie that were Correct}

Use section and subsection commands to organize your document. \LaTeX{} handles all the formatting and numbering automatically. Use ref and label commands for cross-references.

\subsection{Aspects of the Movie that were Incorrect}

You can add inline TODO comments with the todonotes package, like this:
\todo[inline, color=green!40]{This is an inline comment.}

\subsection{Conclusion}

LaTeX automatically generates a bibliography in the APA style from your .bib file. The citep command generates a formatted citation in parentheses \citep{Lamport1986}. The cite command generates one without parentheses. LaTeX was first discovered by \cite{Lamport1986}.

\pagebreak
\maketitle

\subsection{References}
Coppola, F. F. (Producer), \& Coppola, F. F. (Director). (1979). \textit{Apocalypse Now} [Motion picture]. United States: Omni Zoetrope Studios.

American Psychiatric Association. (2013). Diagnostic and statistical manual of mental disorders (5th ed.). Washington, DC: Author.

\end{document}

%
% Please see the package documentation for more information
% on the APA6 document class:
%
% http://www.ctan.org/pkg/apa6
%