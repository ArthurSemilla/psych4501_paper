\documentclass[a4paper,man,natbib]{apa6}

\usepackage{apacite}
\usepackage[english]{babel}
\usepackage[utf8x]{inputenc}
\usepackage{amsmath}
\usepackage{graphicx}
\usepackage[colorinlistoftodos]{todonotes}

\title{A Review of the Movie \textit{Apocalypse Now}}
\shorttitle{A Review of the Movie \textit{Apocalypse Now}}
\author{Samuel Pilla}
\affiliation{Missouri University of Science and Technology}

\abstract{\textit{Apocalypse Now} is a film directed by Francis Frank Coppola. It follows the main character, Captain Benjamin L. Willard, through his last mission in Vietnam. Assigned with assassinating the rogue Captain Walter E. Kurtz, he accompanies four other crew members, Chief, Chef, Clean, and Lance, through the chaos of the Vietnam War. The film dives into the dark progression of these characters, and how the mission Willard is assigned and so desperate to finish affects them and those they encounter, friend or foe. This paper will explore the applications of Abnormal Psychology in the film, including correct and incorrect aspects of symptoms, diagnosis, and other disorders that are depicted in the movie. Using the DSM-V as the source for determining the accuracy of disorders, Willard, Chef, and Lance will be examined to outline which disorder, or disorders, they exhibit and which the film may inaccurately depict. Because the film is based in the Vietnam War, the perspective when analyzing the characters will be shifted to fit the setting, thus potentially causing a diagnosis to be different than the diagnosis of a “normal” life experience.}

\begin{document}
\maketitle

\section{Introduction}

\textit{Apocalypse Now} focuses on Captain Benjamin L. Willard and his last mission during the Vietnam War. The film opens with Willard experiencing images and sounds of the war, hearing the fan as a chopper and watching tree lines being incinerated by air strikes. When he finally snaps out of his trance, he realizes he is not in the jungle, where he longs to be, but rather in a Saigon hotel. He describes his life back home after the first tour as boring, dull, and undesirable. He had divorced his wife, and only wanted to be back in action. While in the hotel room, he slowly falls into madness. Drinking, smoking, wild dancing, punching glass, and other self-inflicted wounds finally lead up to him being called upon for a very particular mission, one he describes as "the last he will ever need".

After being assisted in getting cleaned up and presentable, he meets with some officers and is assigned the mission: assassinate the rogue Colonel Kurtz, who has gone to Cambodia and built his own little empire, being praised as a god, and leading with ruthless authority. Being described as insane and off the rails, Willard is encouraged to kill with extreme prejudice.

As Willard begins his journey towards Kurtz, he reflects on his own actions, on how many "Charlies", or Vietcong, he has killed, and begins to wonder what happened to Kurtz. What about the war really made him snap and go rogue?

He meets up with the Navy crew known as the PBR, consisting of the captain Chief, and three others, Chef, Clean, and Lance. They are assigned to take Willard where he needs to go, however, being that the mission is top secret, the crew is very hesitant. Willard informs them that the only way to reach the destination is through the Nung River, which Chief is somewhat reluctant to go on. The entrances are not always deep enough for the boat, and on top of that are typically crawling with enemies. However, the crew begins to boat towards the Nung River, light-hearted and expected nothing but the expected.

Before reaching the river, they rendezvous with Lieutenant Colonel Bill Kilgore, who is in charge of a helicopter air squadron. They discuss getting to the Nung River, and are met with great resistance until Kilgore discovers Lance is a professional surfer. Being a surfing fan himself, and being informed one of the main entrances to the Nung River has 6 foot waves, he agrees to assist. When arriving at the mouth of the river, Kilgore shows his true colors, blasting "Ride of the Valkyries" while raining death onto the village below. After decimating the village, Kilgore demands two of his men test the waters for surfing, while mortars are bombarding around them from the edges of the jungle around the village. Eventually, a napalm strike on the tree line ends most of the fighting, but begins to end the good surf. Kilgore pleads with Lance to stick around instead of continuing up the Nung River, but Willard craftily gets the PBR crew out, and even manages to steal Kilgore's favorite surf board.

While hiding out from Kilgore's search parties under the low overhang of trees, Chef and Willard decide to wander into the jungle in search of mangoes. After a short walk and discussing the origin of Chef's name, Willard hears rustling and becomes extremely alert. He pulls back some large brush, only to reveal a tiger, which attacks. The two scramble back to the boat, screaming hysterically and forcing Clean to unload the machine gun. When they finally get back on the boat and explain what happened, Chef starts to curse and swear that he is done with the war, demanding to leave. However, he is calmed down while Chief starts the boat and continues on. 

Eventually the crew reaches Hau Phat, a resupply camp they desperately need. After being brushed off when asking for more fuel, Willard takes actions into his own hands. He grabs the soldier in charge of distributing supplies and demands fuel. The soldier agrees and apologizes, also giving the crew free tickets to the Playboy bunnies' show.

The crew is gathered in with hundreds of other US soldiers as the show begins, all raving maniacally to have the women sign their photos, touch them, and eventually swarm the stage, forcing the women to leave. As their helicopter takes off, the stage is swarmed and two soldier cling onto the helicopter rails. Their efforts are futile, and they eventually splash into the water below.

Although only temporarily sated with that distraction from the war and the mission, the crew continues up the river further towards Kurtz. They eventually reach a Medevac camp during a torrential downpour. While trying to find the commanding officer of the camp, Willard stumbles upon the "agent" of the bunnies, and strikes a deal for his crew to spend time with them. While Chef and Lance spend time with two of the bunnies, it becomes apparent the bunnies are not entirely willing to be there. Lance is told by the woman he is spending time with, the Playmate of the Year, she has done things she was very against because its almost a survival instinct. Chef almost entirely ignores the woman he is with, Miss May. As she talks about her love for birds, Chef undresses and poses her to be a mirror of his favorite picture of her from the Playboy calendar. Eventually, he begins to act like a bird, causing her to become sexually aroused and leading to them having sex. After leaving the Medevac camp, Clean complains about not being able to spend time with either bunny, and gets shrugged off by both Chef and Lance. Chef in particular makes fun of Clean and how "he still needs to pop his cherry". Lance is very quiet about his experiences, and instead paint camouflage on his face, saying it will save him from the "Charlies" seeing him.

Further down the river, they stop a trading boat for a routine check for weapons or Vietcong support. Willard demands they do not, but Chief is insistent, telling Willard they have more than just the orders from Willard to follow. Chef is ordered to check the boat, but is extremely resilient. Eventually he does, throwing anything he finds around in a rage. Nothing shows up until a yellow barrel set everyone off guard, causing a Vietnamese woman to frantically yell not to check it. Clean then loses it, and begins firing the machine gun at the trade boat, leading Lance to fire maniacally as well, killing all but the Vietnamese woman. After the chaotic gunfire, all they find is a puppy, which Lance takes as his own. While Chief tells Chef to bring the woman over to bring to a hospital, Willard kills her, reinforcing his previous statement to not stop.

All crew are silent moving on, until they arrive at the last US station, the Do Long Bridge, a bridge constantly being destroyed at night and rebuilt in the day. Willard and Lance begin the search for the commanding officer, slowly realizing the insanity and chaos the men at the bridge have fallen into. Watching the bridge slowly collapse from Vietcong fire, they leave with no fuel or  assistance, and continue the increasingly chaotic path down to Kurtz. Shortly after passing through the Do Long Bridge, Willard learns the old commanding officer, Captain Colby, had been sent on the same mission he is on, but was converted into Kurtz's colony.

Out of nowhere, a barrage of fire consumes the boat, cause panic and return fire. In the heat of the firefight, Clean is downed, and only after does the rest of the crew realize he is dead. 
The combination of all previous stressors from the war and the recent death of Clean drive Lance and Chef to excessive drug use. Devastated by the death of their 17 year old comrade, the sullen crew is stopped in the fog by a mysterious by French troops. After speaking with them, they come to an agreement, letting the US soldiers bury and respectfully say goodbye to Clean. Afterwards, Willard is invited to a dinner, where he learns about the French clan's plantation and how determined they are to stay. After losing World War 1 and 2, these French demand victory even to death. When Willard becomes befuddled and confused why they would be so committed, the French become enraged with Willard and the US Army in general. They point out the US may win every battle, but have no chance to win the war. Nothing can be gained here for the US. This drives everyone else to retire for the evening.

One particular women, Roxanne, stays, and begins connecting with Willard. Roxanne lost her husband, and tells Willard about how she would tell him "There are two of you don't you see? One that kills. One that loves." Willard begins to realize he is exactly that, a man that kills, and a man that loves. However his love had been lost when he left his home and his family, becoming enthralled in the war. Only during his time with Roxanne does he briefly remember his love.

On the final stretch to Kurtz, Chief and Willard flare up at each other, arguing over whether the mission has been worth the losses. Another barrage floods the boat again, except this time of arrows. Chief orders return fire, but after Willard realizes they are just toy arrows, he demands a cease fire. Chief becomes further enraged with Willard's desire for power. As he spews slanders towards Willard, a spear impales him. Coughing up his own blood, he attempts to choke out Willard, a last effort to save his remaining crew, but Willard responds and smothers Chief.

Having lost two men, the remaining crew, Chef, Lance, and Willard, finally arrive at Kurtz's base. They are greeted by a single American photographer and hundreds of Kurtz's followers. The photographer guides Willard and Lance through the colony, Chef staying on the boat, and continually praises Kurtz as a god. The two slowly realize the severity of madness that Kurtz has descended into, ruthlessly murdering anyone who defies his reign and mutilating the bodies of fallen enemies. While exploring the camp, they come across the practically brainwashed Colby, the ex-commanding officer of the DO Long Bridge and last man sent to assassinate Kurtz. As the new servant stares down Willard and Chef, they realize the extreme devotion these people have for Kurtz.

They return from the boat again, this time Willard and Lance, to talk to Kurtz, telling Chef to call an airstrike if they do not return. While finally meeting Kurtz in person, Willard is captured and imprisoned, while Lance, having totally lost touch with himself and reality, becomes one of Kurtz's men. Willard, now beaten and tied to a pole, is visited by Kurtz, who drops the head of Chef on his lap. Willard realizes Chef had not gotten the air strike called in and becomes distraught, crying out for the mistakes he has made.

Eventually he is set free, under prohibition, wandering the colony Kurtz has built. While under watch, Kurtz lectures Willard on the reality of war and humanity, reflecting the French plantation words that US can not win with the dedication of the Vietcong.

That night the colony throws a celebration, so Willard takes the opportunity to sneak into Kurtz quarters. As Kurtz reads off quotes to be sent to the US army, Willard attacks him with a machete, eventually forcing Kurtz to the ground. As Kurtz breaths his final breaths, he utters the words "The horror...the horror...". As Willard leaves from Kurtz's quarters, praised by the now aware colony as the new god, he grabs Lance to leave. The film finishes with the two leaving the colony  in silence, begging the question of who is really insane in war, particularly when it can not be won.

\section{Applications to Abnormal Psychology}

The film portrays the psychological effects have on many characters, prominently Willard, Chef, and Lance.

Willard exhibits many symptoms throughout the film, many of which fall under the diagnosis of Post-Traumatic Stress Disorder. Using the DSM-V, the following symptoms can be evaluated as such (American Psychiatric Association, 2013):
\begin{itemize}
\item Previous Vietnam tour, indicative of potential exposure to death or major injury.
\item Vivid dream of previous combat, or repeated, intrusive exposure.
\item Divorce after returning to the US, having said "The only words I spoke to my wife were 'Yes' when she asked for a divorce", showing both little interest in important events and feelings of detachment.
\item Excessive drinking, smoking, punching a mirror, and other self harm, exhibiting irritable and reckless, self-harming behaviors.
\item Has been a changed person since the previous tour, over 1 month ago.
\end{itemize} 
Willard appears to be another not uncommon case of a soldier returning from war with PTSD, however there a few missing criteria from the DSM-V that Willard fails to show, or are the opposite of symptoms for the diagnosis (American Psychiatric Association, 2013):
\begin{itemize}
\item Willard desires to be back "in the jungle", or back on a mission, which would counter that the original trauma was during his first tour.
\item Willard has no exaggerated or distorted memories of the previous tour.
\item Willard wants to return, which shows a lack of avoidance behaviors.
\item Willard experiences happiness, as well as interacting with the crew well.
\end{itemize}
The first three points would prevent a diagnosis of PTSD for Willard, while the last would not necessarily effect the diagnosis either way, since it is part of a subset of symptoms that Willard exhibits (American Psychiatric Association, 2013). His obsession with returning to the jungle could be though as part of Obsessive-Compulsive Disorder, however the obsession is not unwanted, not causing stress, and there are no cleansing compulsions to reduce or get rid of the obsession (American Psychiatric Association, 2013).

Chef seems a rather normal man by military standards, interacting with his crew, coping with the situations he has faced rather well. However, the hellish mission he embarks on with Willard causes this to change. As they continue further up the Nung River, Chef begins to develop symptoms of Acute Stress Disorder as follows (American Psychiatric Association, 2013):
\begin{itemize}
\item Chef is attacked by a tiger, having a near death experience. He also witnesses the death of 17 year old Clean, another traumatic event.
\item While searching the boat, he is extremely resistance to begin the search and becomes enraged while searching, showing avoidance and arousal symptoms
\item Chef consistently disagrees with Willard's mission, and questions why they are helping him anyways, exhibiting persistent negative behavior
\end{itemize}
Although he also has symptoms of a substance abuse disorder, specifically when he and Lance start excessive use after Clean's death, this is a response to the stress, and thus would not be Substance Use Disorder, rather Acute Stress Disorder with a response of substance use. However, even with these present, there is a definitive lack of and dissociative symptoms. He also exhibits no intrusion symptoms since no flashbacks are present, and the intense and prolonged stress is not caused from the traumatic events specifically, but are rather the stressful events in the moment(American Psychiatric Association, 2013). 

Lance begins the film a very naive, California surfer, almost out of place in the Vietnam War. As more and more significantly stressful events occur, he relapses into a shell, showing signs of Major Depressive Disorder (American Psychiatric Association, 2013):
\begin{itemize}
\item Lance shows significant physical lethargy, or psychomotor retardation.
\item Lance does not take interest in many activities, even ones outside the mission
\item Along with the lethargy, Lance appears exhausted even when rested outside the stressful combat and other events
\item Lance does not appear comfortable in any situation outside his times with the puppy
\item There is no known excessive substance use to cause the distress
\end{itemize}
However, there are not enough shown symptoms to diagnose Lance with Major Depressive Disorder. Instead,he might be experiencing Depressive episode with insufficient symptoms. This would be a more accurate diagnosis with the known symptoms in the film, as well as the lack of any manic episodes or other bipolar specific symptoms (American Psychiatric Association, 2013). He could also be diagnosed with PTSD, using the death of Clean and Chief as the trauma. Similarly to Chef, he could be diagnosed with a substance use response, since the substance use only appeared after the original symptoms for depression or PTSD.

\section{Conclusion}	

According to the DSM-V, Willard is the only person who could get diagnosed with a disorder, in his case PTSD. Chef could possibly have more symptoms, however the film does not portray them. Further, he could have an Unspecified Trauma- and
Stressor-Related Disorder, since his symptoms are not definitive of Acute Stress Disorder or another specific stress disorder. Lance could be diagnosed with Major Depressive Disorder, although it would require more symptoms to be present. With the films portrayal, it is more likely he is diagnosed with a Depressive episode lacking sufficient symptoms or PTSD with catatonic behaviors, since he changed his behaviors after the death of Clean. Lance and Chef could both also have an Substance Use added as a response to their potential diagnosis. \\

\raggedbottom
\pagebreak
\section{References}

\noindent American Psychiatric Association. (2013). Diagnostic and statistical manual of mental
\indent disorders (5th ed.). Washington, DC: Author.

\noindent Coppola, F. F. (Producer), \& Coppola, F. F. (Director). (1979). \textit{Apocalypse Now}
\indent [Motion picture]. United States: Omni Zoetrope Studios.


\end{document}