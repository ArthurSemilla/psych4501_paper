\documentclass[a4paper,man,natbib]{apa6}

\usepackage{apacite}
\usepackage[english]{babel}
\usepackage[utf8x]{inputenc}
\usepackage{amsmath}
\usepackage{graphicx}
\usepackage[colorinlistoftodos]{todonotes}

\shorttitle{A Review of the Movie \textit{Apocalypse Now}}

\title{A Review of the Movie \textit{Apocalypse Now}}
\author{Samuel Pilla}
\affiliation{Missouri University of Science and Technology}

\abstract{\textit{Apocalypse Now} is a film directed by Francis Frank Coppola. It follows the main character, Captain Benjamin L. Willard, through his last mission in Vietnam. Assigned with assassinating the rogue Captain Walter E. Kurtz, he is accompanied by four other crew members, Chief, Chef, Clean, and Lance, through the chaos of the Vietnam War. The film dives into the dark progression of these characters and how the mission Willard was desperate to finish affected them and those they encountered, friend or foe. This paper will explore the applications of Abnormal Psychology in the film, including correct and incorrect aspects of symptoms, diagnosis, and other disorders that are depicted in the film. Using the DSM-V as the source for determining the accuracy of disorders, Willard, Chef, and Lance will be examined to outline which disorder, or disorders, they exhibit and which the film may inaccurately depict. Because the film is based in the Vietnam War, the perspective when analyzing the characters will be shifted to fit the setting, potentially causing a diagnosis to be different than the diagnosis of a civilian life experience.}

\begin{document}
\maketitle

\textit{Apocalypse Now} focuses on Captain Benjamin L. Willard and his last mission during the Vietnam War. The film opens with Willard experiencing images and sounds of the war, hearing the fan of a chopper, and watching tree lines be incinerated by air strikes. When he finally snapped out of his trance, he realized he is not in the jungle, where he longed to be, but rather in a Saigon hotel. He described his life back home after the first tour as boring, dull, and undesirable. He had divorced his wife, and only wanted to be back in action. While in the hotel room, he slowly fell into madness. Drinking, smoking, wild dancing, punching glass, and other self-inflicted wounds finally led up to him being called upon for a very particular mission, one he described as "the last he will ever need" (Coppola, 1979).

After being assisted in getting cleaned up and presentable, Willard met with his superior officers. He wad then assigned the mission: assassinate the rogue Colonel Kurtz, who had gone to Cambodia and built his own empire. Kurtz was being praised as a god, and led with ruthless authority. Described as insane and off the rails, Willard is encouraged to kill with extreme prejudice.

As Willard began his journey towards Kurtz, he reflected on his own actions, on how many "Charlies", or Vietcong, he had killed, and began to wonder what happened to Kurtz (Coppola, 1979). What about the war really made him snap and go rogue?

He met up with the Navy crew known as the PBR, consisting of the captain Chief, and three others, Chef, Clean, and Lance. They are assigned to take Willard where he needs to go. However, the mission was top secret, which made the crew very hesitant. Willard informed them that the only way to reach the destination was through the Nung River, which Chief was somewhat reluctant to go on. The entrances were not always deep enough for the boat, and were typically crawling with enemies. However, the crew began to boat towards the Nung River, filled with light hearts.

Before reaching the river, they rendezvoused with Lieutenant Colonel Bill Kilgore, who was in charge of a helicopter air squadron. They discussed getting to the Nung River, and are met with great resistance until Kilgore discovered Lance was a professional surfer. A surfing fan himself, and being informed one of the main entrances to the Nung River had six foot waves, he agreed to assist. When they arrived at the mouth of the river, Kilgore showed his true colors, blasting "Ride of the Valkyries" while raining death onto the village below. After decimating the village, Kilgore demanded two of his men test the waters for surfing, while mortars bombarded around them from the edges of the jungle. Eventually, a napalm strike on the tree line ended most of the fighting, but began to end the good surf. Kilgore pleaded with Lance to stick around instead of continuing up the Nung River, but Willard craftily got the PBR crew out, and even managed to steal Kilgore's favorite surf board.

While hiding out from Kilgore's search parties under the low overhang of trees, Chef and Willard decided to wander into the jungle in search of mangoes. After a short walk and discussing the origin of Chef's name, Willard heard rustling and became extremely alert. He pulled back some large brush, only to reveal a tiger, which attacked. The two scrambled back to the boat, screaming hysterically and forcing Clean to unload the machine gun. When they finally get back on the boat and explained what happened, Chef started to curse and swear that he is done with the war, demanding to leave. However, he was calmed down while Chief started the boat and continued on. 

Eventually the crew reached Hau Phat, a resupply camp they desperately needed. After being brushed off when asking for more fuel, Willard took actions into his own hands. He grabbed the soldier in charge of distributing supplies and demanded fuel. The soldier agreed and apologized, also giving the crew free tickets to the Playboy bunnies' show.

The crew is gathered with hundreds of other US soldiers as the show began, all of them raving maniacally to have the women sign their photos and touch them. They eventually swarmed the stage, forcing the women to leave. As their helicopter took off, two soldiers clung onto the helicopter rails. Their efforts were futile, and they eventually splashed into the water below.

Although only temporarily sated with that distraction from the war and the mission, the crew continued up the river further towards Kurtz. They eventually reached a Medevac camp during a torrential downpour. While trying to find the commanding officer of the camp, Willard stumbled upon the pimp of the Playboy models from the show, and struck a deal for his crew to spend time with the women. While Chef and Lance spent time with two of the bunnies, it became apparent the bunnies are not entirely willing to be there. Lance was told by the woman he was spending time with, the Playmate of the Year, she had done things that she was very against because it was almost a survival instinct. Chef ignored the woman he was with, Miss May. As she talked about her love for birds, Chef undressed and posed her to be a mirror of his favorite picture of her from the Playboy calendar. Eventually, he began to act like a bird, causing her to become sexually aroused and leading to them having sex. After leaving the Medevac camp, Clean complained about not being able to spend time with either bunny, but got shrugged off by both Chef and Lance. Chef in particular made fun of Clean and how "he still needs to pop his cherry" (Coppola, 1979). Lance was very quiet about his experience, and instead painted camouflage on his face, saying it will save him from the "Charlies" seeing him (Coppola, 1979).

Further down the river, they stopped a trading boat for a routine weapons check for Vietcong support. Willard demanded they do not perform the check, but Chief was insistent, telling Willard they have more than just the orders from Willard to follow. Chef was ordered to check the boat, but was extremely resilient. Eventually he does, throwing anything he found around in a rage. Nothing showed up until a yellow barrel set everyone off guard, which caused a Vietnamese woman to frantically yell not to check it. Clean then lost it, and began firing the machine gun at the trade boat, leading Lance to fire maniacally as well, killing all but the Vietnamese woman. After the chaotic gunfire, all they found was a puppy, which Lance took as his own. While Chief told Chef to bring the woman to a hospital, Willard killed her, reinforcing his previous statement to not stop.

All crew were silent moving on, until they arrived at the last US station, the Do Long Bridge, a bridge constantly being destroyed at night and rebuilt in the day. Willard and Lance began the search for the commanding officer, slowly realizing the insanity and chaos the men at the bridge have fallen. Watching the bridge slowly collapse from Vietcong fire, they left with no fuel or assistance, and continued the increasingly chaotic path down to Kurtz. Shortly after passing through the Do Long Bridge, Willard learned the old commanding officer, Captain Colby, had been sent on the same mission he is on, but was converted into Kurtz's colony.

Out of nowhere, a barrage of fire consumed the boat, causing panic and returned fire. In the heat of the firefight, Clean is downed, and only afterwards did the rest of the crew realize he is dead. 
The combination of all previous stressors from the war and the recent death of Clean drove Lance and Chef to excessive drug use. Devastated by the death of their 17 year old comrade, the sullen crew is stopped in the fog by mysterious French troops. After speaking with them, they came to an agreement, letting the US soldiers bury and respectfully say goodbye to Clean. Afterwards, Willard was invited to a dinner, where he learned about the French clan's plantation and how determined they were to stay. After losing World War I and II, the French demanded victory even to death. When Willard became befuddled and confused why they would be so committed, the French became enraged with Willard and the US Army in general. They pointed out the US may have won every battle, but had no chance to win the war. Nothing could be gained there for the US. This powerful statement drove everyone else to retire for the evening.

One particular women, Roxanne, stayed, and began connecting with Willard. Roxanne lost her husband, and told Willard about how she would tell him "There are two of you don't you see? One that kills. One that loves." (Coppola, 1979). Willard began to realize he is exactly that, a man that killed, and a man that loved. However, his love had been lost when he left his home and his family, becoming enthralled in the war. Only during his time with Roxanne did he briefly remember his love.

On the final stretch to Kurtz, Chief and Willard flared up at each other, arguing over whether the mission had been worth the losses. A barrage of arrows flooded the boat, and Chief ordered return fire. After Willard realized they are just toy arrows, he demanded a cease fire. Chief became further enraged with Willard's desire for power. As he spewed slanders towards Willard, a spear impaled him. Coughing up his own blood, he attempted to choke out Willard, a last effort to save his remaining crew, but Willard responded and smothered Chief.

Having lost two men, the remaining crew, Chef, Lance, and Willard, finally arrived at Kurtz's base. They are greeted by a single American photographer and hundreds of Kurtz's followers. The photographer guided Willard and Lance through the colony and continually praised Kurtz as a god. The two slowly realized the severity of madness that Kurtz had descended into, ruthlessly murdering anyone who defied his reign and mutilating the bodies of fallen enemies. While exploring the camp, they came across the practically brainwashed Colby, the ex-commanding officer of the Do Long Bridge and last man sent to assassinate Kurtz. As the new servant stared down Willard and Chef, they realized the extreme devotion these people had for Kurtz.

Willard and Lance returned to the village to talk to Kurtz, telling Chef to call an airstrike if they did not come back. While finally meeting Kurtz in person, Willard is captured and imprisoned, while Lance, having totally lost touch with himself and reality, became one of Kurtz's men. Willard, now beaten and tied to a pole, was visited by Kurtz, who dropped the head of Chef on his lap. Willard realized Chef had not gotten the air strike called in and became distraught, crying out for the mistakes he had made.

Eventually he was set free, under prohibition, wandering the colony Kurtz built. While under watch, Kurtz lectured Willard on the reality of war and humanity, reflecting the French plantation words that US could not win with the dedication of the Vietcong.

That night the colony threw a celebration, so Willard took the opportunity to sneak into Kurtz quarters. As Kurtz read off quotes to be sent to the US army, Willard attacked him with a machete, eventually forcing Kurtz to the ground. As Kurtz breathes his final breaths, he uttered the words "The horror...the horror..." (Coppola, 1979). As Willard left from Kurtz's quarters, praised by the colony, who were aware Kurtz had been killed, he grabs Lance to leave. The film finished with the two leaving the colony  in silence, begging the question of who is really insane in war, particularly when it can not be won.

\section{Applications to Abnormal Psychology}

The film portrayed the psychological effects war had on many characters, prominently Willard, Chef, and Lance.

Willard exhibited many symptoms throughout the film, many of which fall under the diagnosis of Posttraumatic Stress Disorder (PTSD). The following symptoms of PTSD are exhibited by Willard (American Psychiatric Association, 2013):
\begin{itemize}
\item Previous Vietnam tour, indicative of potential exposure to death or major injury.
\item Vivid dream of previous combat, or repeated, intrusive exposure.
\item Divorce after returning to the US, having said "The only words I spoke to my wife were 'Yes' when she asked for a divorce", showing both little interest in important events and feelings of detachment (Coppola, 1979).
\item Excessive drinking, smoking, punching a mirror, and other self harm acts, exhibiting irritable, reckless, and self-harming behaviors.
\item Has been a changed person since the previous tour, over one month ago.
\end{itemize} 
Willard appears to be another not uncommon case of a soldier returning from war with PTSD; however, there are few missing criteria from the DSM-V that Willard failed to show, or are the opposite of symptoms for the diagnosis (American Psychiatric Association, 2013):
\begin{itemize}
\item Willard desires to be back "in the jungle", or back on a mission, which would counter that the original trauma was during his first tour (Coppola, 1979).
\item Willard has no exaggerated or distorted memories of the previous tour.
\item Willard wants to return, which shows a lack of avoidance behaviors.
\item Willard experiences happiness, as well as interacting with the crew well.
\end{itemize}
The first three points would prevent a diagnosis of PTSD for Willard, while the last would not necessarily effect the diagnosis either way, since it is part of a subset of symptoms that Willard exhibited (American Psychiatric Association, 2013). His obsession with returning to the jungle could exhibit symptoms of Obsessive Compulsive Disorder, but the obsession was not unwanted, not causing stress, and there were no cleansing compulsions to reduce or get rid of the obsession (American Psychiatric Association, 2013).

Chef seemed a rather normal man by military standards, interacting with his crew, coping with the situations he had faced rather well. However, the hellish mission he embarked on with Willard caused his mental state to change. As they continued further up the Nung River, Chef began to develop symptoms of Acute Stress Disorder as follows (American Psychiatric Association, 2013):
\begin{itemize}
\item Chef is attacked by a tiger, having a near death experience. He also witnesses the death of 17 year old Clean, another traumatic event.
\item While searching the boat, he is extremely resistance to begin the search and becomes enraged while searching, showing avoidance and arousal symptoms
\item Chef consistently disagrees with Willard's mission, and questions why they are helping him anyways, exhibiting persistent negative behavior
\end{itemize}
Although he also had symptoms of a Substance Use Disorder, specifically when he and Lance started excessive drug use after Clean's death, this was a response to the stress, and thus would not be Substance Use Disorder, but rather Acute Stress Disorder with a response of substance use. But even with these symptoms present, there was a definitive lack of any dissociative symptoms. He also exhibited no intrusion symptoms since no flashbacks are present, and the intense and prolonged stress was not caused from the traumatic events specifically, but were rather the stressful events in the moment (American Psychiatric Association, 2013). 

Lance began the film a very naive, California surfer, almost out of place in the Vietnam War. As more significantly stressful events occurred, he relapsed into a shell, showing signs of Major Depressive Disorder (American Psychiatric Association, 2013):
\begin{itemize}
\item Lance shows significant physical lethargy, or psychomotor retardation.
\item Lance does not take interest in many activities, even ones outside the mission
\item Along with the lethargy, Lance appears exhausted even when rested outside the stressful combat and other events
\item Lance does not appear comfortable in any situation outside his times with the puppy
\item There is no known excessive substance use to cause the distress
\end{itemize}
However, there are not enough symptoms to diagnose Lance with Major Depressive Disorder. Instead, he might have experienced a depressive episode with insufficient symptoms. This diagnosis would be more accurate given the symptoms exhibited in the film, as well as the lack of any manic episodes or other bipolar specific symptoms (American Psychiatric Association, 2013). He could also be diagnosed with PTSD, using the death of Clean and Chief as the trauma. Similarly to Chef, he could be diagnosed with a substance use response, since the substance use only appeared after the original symptoms for depression or PTSD.

\section{Conclusion}	

According to the DSM-V, Willard was the only person who met the diagnosis for PTSD. Chef could have possibly had more symptoms, however, the film did not portray them. Further, he could have an Unspecified Trauma and Stressor-Related Disorder, since his lack of symptoms were not definitive of Acute Stress Disorder or another specific stress disorder. Lance could be diagnosed with Major Depressive Disorder, although it would require more symptoms to have be present. With the films portrayal, it is more likely he would be diagnosed with a depressive episode lacking sufficient symptoms or PTSD with catatonic behaviors, since he changed his behaviors after the death of Clean. Lance and Chef could both also have a Substance Use Disorder added as a response to their potential diagnosis. \\

\raggedbottom
\pagebreak
\section{References}

\noindent American Psychiatric Association. (2013). Diagnostic and statistical manual of mental
\indent disorders (5th ed.). Washington, DC: Author.

\noindent Coppola, F. F. (Producer), \& Coppola, F. F. (Director). (1979). \textit{Apocalypse Now}
\indent [Motion picture]. United States: Omni Zoetrope Studios.


\end{document}